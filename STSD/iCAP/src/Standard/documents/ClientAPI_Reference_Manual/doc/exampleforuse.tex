\section{Example for use}

\subsection{GNC-C sample code}

\begin{lstlisting}
#include <stdio.h>
#include <stdlib.h>
#include <unistd.h>
#include <signal.h>
#ifdef WIN32
#include <windows.h>
#else
#include <sys/wait.h>
#endif
#include "lib_iCAPClient.h"

char onExit = 0;

#ifdef WIN32
BOOL WINAPI gotExitCmd(DWORD dwType)
{
    switch(dwType) {
		case CTRL_C_EVENT:
		case CTRL_CLOSE_EVENT:
			onExit = 1;
			break;
    }
    return TRUE;
}
#else
void gotExitCmd(int sig)
{
	onExit = 1;
}

void gotPAUSECmd(int sig)
{
	printf("Send event trigger to client service\n");
	iCAP_EventTrigger();
}
#endif

double GetSensorValue(void)
{
	return 25.5;
}

int ReceiveRemoteCmd(void* cmd)
{
	char* cmd_str = (char*)cmd;
	printf("Get remote command : %s\n", cmd_str);
	return 0;
}

int main(int argc, char** argv)
{
	int ret;
	char c;

#ifdef WIN32
	if(!SetConsoleCtrlHandler((PHANDLER_ROUTINE)gotExitCmd,TRUE))
	{
		printf("Fail to listen Ctrl-C signal!\n");
	}
#else
	(void)signal(SIGINT, gotExitCmd);
	(void)signal(SIGTSTP, gotPAUSECmd);
#endif

	if(iCAP_Connect() == 0)
	{
		printf("Connect to iCAP client service successfully\n");
		
		ret = iCAP_GetClientStatus();
		printf("Device servcie status : %d\n", ret);

		iCAP_AddExternalSensor("Temp", "c", 0, GetSensorValue);
		iCAP_AddRemoteDevice("TestRemote", "v", 0, ReceiveRemoteCmd);

		while(onExit == 0)
		{
			usleep(10000L);
		}

		iCAP_RemoveExternalSensor("Temp");
		iCAP_RemoveRemoteDevice("TestRemote");

		iCAP_Disconnect();
	}
	else
	{
		printf("Connect to iCAP client service fail\n");
	}
	
	return 0;
}

\end{lstlisting}